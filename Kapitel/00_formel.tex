\begin{table}[h!]

\centering
\caption{Allgemeine Parameter}
\label{tab:parameter}
\renewcommand{\arraystretch}{2}
\setlength{\tabcolsep}{10mm}

\begin{tabular}{lll}
    \toprule
     Bezeichnung Python & Bedeutung & Formelzeichen\\
    \midrule
	num\textunderscore pole\textunderscore pairs&-&$p$\\
	num\textunderscore coils &-&$n_{coil}$\\
	rot\textunderscore speed&-&$n_{rotor}$\\
	M\textunderscore T&&$M_{T}$\\
	R\textunderscore L&Lastwiderstand&$R_L$\\
	b\textunderscore avg&durch. Mag.feld&$b_{avg}$\\
	angle\textunderscore magnet&Bogenlänge zw. Magnet&$l_ {mag}$\\
	angle\textunderscore magnet\textunderscore space&Bogenlänge Magnet&$l_ {mag.space}$\\
	angle\textunderscore coil&Bogenlänge Spule&$l_{coil}$\\
	angle\textunderscore coil\textunderscore space&&$l_{coil.space}$\\
	rotor\textunderscore r\textunderscore inner&Radius zu Mag. Innen&$r$\\
	&ddffd&$$\\
	&ddffd&$f_i$\\
	
    \bottomrule
  \end{tabular}
\end{table}

\begin{table}[h!]

\centering
\caption{Formlen aus der Class (Radial)}
\label{tab:class.rad}
\renewcommand{\arraystretch}{2}
\setlength{\tabcolsep}{10mm}

\begin{tabular}{ll}
    \toprule
     Bezeichnung & Formel\\
    \midrule
	Platz zwischen Magneten&$\frac{}{}$\\
	Platz zwischen Magneten&$\frac{}{}$\\
	
    \bottomrule
  \end{tabular}
\end{table}







