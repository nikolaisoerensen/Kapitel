\begin{table}[h!]

\centering
\caption{Allgemeine Konstanten}
\label{tab:parameter}
\renewcommand{\arraystretch}{2}
\setlength{\tabcolsep}{7mm}

\begin{tabular}{lll}
    \toprule
     Bezeichnung Python & Bedeutung & Formelzeichen\\
    \midrule
	num\_pole\_pairs &-&$p = 4$\\
	num\_coils &-&$n_{coil} = 4$\\
	rot\_speed &-&$n_{rotor}$\\
	M\_T &&$M_{T}$\\
	R\_L &Lastwiderstand&$R_L$\\

    \bottomrule
  \end{tabular}
\end{table}

\newpage

%&&$$\\

\begin{table}[h!]

\centering
\caption{Konstanten Radialgenerator}
\label{tab:const_rad}
\renewcommand{\arraystretch}{2}
\setlength{\tabcolsep}{5mm}

\begin{tabular}{lll}
    \toprule
     Bezeichnung Python & Bedeutung & Formelzeichen und Wert\\
    \midrule
    b\_avg &durch. Mag.feld &$b_{avg}$\\
	angle\_magnet &Bogenlänge Magnet&$\alpha_ {mag}=70\degree$\\
	angle\_coil &Bogenlänge Spule&$\alpha_{coil} = 20\degree$\\
	rotor\_r\_inner &Radius zu Mag. Innen&$r_{rot.in} = 35 \text{mm}$\\
	rotor\_r\_outer &Radius zu Mag. Innen&$r_{rot.out} = 45 \text{mm}$\\
	stator\_r\_inner &Radius zu Stat. Innen&$r_{stat.in} = 47 \text{mm}$\\
	stator\_r\_outer &Radius zu Stat. Innen&$r_{stat.out} = 50 \text{mm}$\\
  l\_coil\_eff &effektive Länge&$l_{coil.eff} = 120 \; \text{mm} $\\


    \bottomrule
  \end{tabular}
\end{table}

\begin{table}[h!]

\centering
\caption{Funktionen aus der Klasse (Radial)}
\label{tab:class.rad}
\renewcommand{\arraystretch}{2}
\setlength{\tabcolsep}{5mm}

\begin{tabular}{lll}
    \toprule
     Bezeichnung Python & Bedeutung& Formel\\
    \midrule
	angle\_space &Bogenlänge zw. Magnet&$\alpha_ {mag.space} = \frac{180}{p} - \alpha_{mag}$\\
	angle\_coil\_space &Bogenlänge zw. Spule&$\alpha_{coil.space} = \frac{360}{p} -  \alpha_{coil}$\\
	r\_magnet &Rad. Mag. innen&$r_{mag} = \frac{r_{rot.in} + r_{rot.out}}{2}$\\
  dist\_rot\_stat &Spaltgröße&$l_{spalt}= 2\; \text{mm} + r_{stat.out} - r_{stat.in}$\\
  l\_coil\_outer &-&$l_{coil.out}=\frac{r_{stat.out} \cdot 2 \cdot \pi }{360 \degree} \cdot (\alpha_{coil} + \alpha_{coil.space})$\\
  l\_coil\_inner &-&$l_{coil.in}=\frac{r_{stat.out} \cdot 2 \cdot \pi }{360 \degree} \cdot (\alpha_{coil} + \alpha_{coil.space})$\\
  l\_coil\_space &-&$l_{coil.space}= \frac{2 \cdot r_{stat.out} \cdot \pi \cdot \alpha_{coil.space}}{360 \degree}$\\
 
	

    \bottomrule
  \end{tabular}
\end{table}

\newpage

\begin{table}[h!]

\centering
\caption{Konstanten Axial}
\label{tab:const_rad}
\renewcommand{\arraystretch}{2}
\setlength{\tabcolsep}{5mm}

\begin{tabular}{lll}
    \toprule
     Bezeichnung Python & Bedeutung & Formelzeichen und Wert\\
    \midrule
    b\_avg &durch. Mag.feld &$b_{avg}$\\
  angle\_magnet &Bogenlänge Magnet&$\alpha_ {mag}=60\degree$\\
  angle\_coil &Bogenlänge Spule&$\alpha_{coil} = 20\degree$\\
  rotor\_r\_inner &Radius zu Mag. Innen&$r_{rot.in} = 45,5 \text{mm}$\\
  rotor\_r\_outer &Radius zu Mag. Innen&$r_{rot.out} = 90,5 \text{mm}$\\
  stator\_r\_inner &Radius zu Stat. Innen&$r_{stat.in} = 45,5 \text{mm}$\\
  stator\_r\_outer &Radius zu Stat. Innen&$r_{stat.out} = 90,5 \text{mm}$\\
  dist\_rot\_stat &Spaltgröße&$l_{spalt}= 1\; \text{mm}$\\
  l\_coil\_eff &effektive Länge&$l_{coil.eff} = 45 \; \text{mm} $\\


    \bottomrule
  \end{tabular}
\end{table}



\begin{table}[h!]

\centering
\caption{Funktionen aus der Klasse (Axial)}
\label{tab:class.axial}
\renewcommand{\arraystretch}{2}
\setlength{\tabcolsep}{5mm}

\begin{tabular}{lll}
    \toprule
     Bezeichnung Python & Bedeutung& Formel\\
    \midrule
 
  angle\_space &Bogenlänge zw. Spule&$\alpha_{space} = \frac{180}{p} -  \alpha_{coil}$\\
  r\_magnet &Rad. Mag. innen& $ r_{mag} = \frac{r_{rot.in} + r_{rot.out}}{2} $\\
  l\_coil\_outer &-&$ l_{coil.out}=\frac{r_{stat.out} \cdot 2 \cdot \pi }{360 \degree} \cdot (\alpha_{magnet} + \alpha_{space})$\\
  l\_coil\_inner &-&$ l_{coil.in}=\frac{r_{stat.in} \cdot 2 \cdot \pi }{360 \degree} \cdot (\alpha_{magnet} + \alpha_{space})$\\
  l\_coil\_space &-&$ l_{coil.space}= \frac{(r_{rot.in} + r_{rot.out}) \cdot \pi \cdot \alpha_{space}}{360 \degree}$\\
  max\_coil\_width&maximale Spulenweite&$l_{coil.width.max} = l_{coil.in} \cdot 0,8$\\
 
  

    \bottomrule
  \end{tabular}
\end{table}

\newpage  

\begin{table}[h!]

\centering
\caption{Konstanten aus Torque.py}
\label{tab:cont_Torque}
\renewcommand{\arraystretch}{2}
\setlength{\tabcolsep}{7mm}

\begin{tabular}{lll}
    \toprule
     Bezeichnung Python & Bedeutung & Formelzeichen\\
    \midrule
    rho &-&$\rho   = 1,224\; \text{bar}$\\
    Turb\_n &-&$n_{turb} = \text{Datei}$\\
    Turb\_M &-&$M_{turb} = \text{Datei}$\\
    v &Windgeschw.&$10\; \frac{m}{s}$\\
    r&??&$450 \cdot 10^{-3}$\\
 

    \bottomrule
  \end{tabular}
\end{table}

\begin{table}[h!]

\centering
\caption{Formeln aus Torque.py}
\label{tab:parameter_Torque}
\renewcommand{\arraystretch}{2}
\setlength{\tabcolsep}{7mm}

\begin{tabular}{lll}
    \toprule
     Bezeichnung Python & Bedeutung & Formelzeichen\\
    \midrule
    P\_wind &-&$P_{Wind}   = \frac{1}{2}\cdot \rho  \cdot v^3 \cdot \pi \cdot r^2 $\\
    P\_Rotor &-&$P_{Rotor}   = 2 \cdot \pi \cdot n_ {turb} \cdot M_{turb}$\\
    Turb\_M &-&$M_{turb} = \text{Datei}$\\ 
    cp &&$\eta  = \frac{P_{Wind}}{P_{Rotor}}$\\
 

    \bottomrule
  \end{tabular}
\end{table} 