Für den Bau des Generators haben wir zwei verschiedene Bauarten betrachet, Axial sowie Radial. Da die Auswahl dieser erst später anhand der Effizienz erfolgt werden zuvor die Geometriedaten bestimmt. blabla \colorbox{yellow}{Hier fehlt noch was}
\begin{itemize}
	\item Läufer gegeben $\rightarrow$ deswegen erstmal nur Daten für den Läufer
	\item anhand der Geometrie kann Mag.feld bestimmt werden 
	\item mit dem Mag.feld kann eine Auswahl für die Geometrie des Stators getroffen werden
\end{itemize}

\subsection{Radial-Läufer}

\begin{table}[h!]

\centering
\caption{Maße des Radial-Läufers}
\label{tab:radial_lauf}
\renewcommand{\arraystretch}{2}
\setlength{\tabcolsep}{7mm}

\begin{tabular}{lll}
    \toprule
     Bedeutung & Bezeichung & Wert\\
    \midrule

	Bogenlänge Magnet&$\alpha_ {mag}$&$70\degree$\\
	Bogenlänge zwischen Magnet&$\alpha_ {mag.space}$& $25 \degree $ \\
	Radius zu Magnet innen&$r_{rot.in} $&$ 35 \text{mm}$\\
	Radius zu Magnet außen&$r_{rot.out}$&$ 45 \text{mm}$\\
	Radius Mitte Magnet &$r_{mag.mid}$&$40\; \text{mm}$\\

    \bottomrule
  \end{tabular}
\end{table}

\subsection{Axial-Läufer}

\begin{table}[h!]

\centering
\caption{Maße des Axial-Läufers}
\label{tab:axial_lauf}
\renewcommand{\arraystretch}{2}
\setlength{\tabcolsep}{7mm}

\begin{tabular}{lll}
    \toprule
     Bedeutung & Bezeichung & Wert\\
    \midrule

	Bogenlänge Magnet&$\alpha_ {mag}$&$60\degree$\\
	Bogenlänge zwischen Magnet&$\alpha_ {mag.space}$& $15 \degree $ \\
	Radius zu Magnet innen&$r_{rot.in} $&$ 45,5 \text{mm}$\\
	Radius zu Magnet außen&$r_{rot.out}$&$ 90,5 \text{mm}$\\
	Radius Mitte Magnet &$r_{mag.mid}$&$68\; \text{mm}$\\

    \bottomrule
  \end{tabular}
\end{table}






