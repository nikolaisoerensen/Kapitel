

Zunächst wird die Klasse Generator erstellt und als Axialgenerator mit zwei Polpaaren und vier Spulen initiiert. Dann werden einige Klasseneigenschaften erschaffen und ihnen werden Werte zugewiesen. So wird zum Beispiel der Eigenschaft „design“ der Startwert „axial“ übergeben. 



Vorgaben durch die Wettbewerbsleitung:

Maße des Magneten: 	L x B x T: 120 x20 x10 mm Radialgenerator
				T x ra x ri: 10 x 95 x 48 mm Axialgenerator

Bezeichnung und ihre Bedeutung:

„Generator.*“:
design: 		      Zeigt an ob es sich um einen Axial- oder Radialgenerator handelt.
num_pole_pairs: 	Gibt die Zahl der Polpaare wider
num_coils: 		    Anzahl der Spulen auf dem Stator
rot_speed:		    Nenndrehzahl des Rotors von Rotorgruppe (24,76 m/s)
M_T:			        Drehmoment bei Nenndrehzahl (1,1 Nm)
R_L:			        Lastwiderstand 					??
b_avg:		        B-Feld soll aus Liste gelesen werden, die mit Hilfe von FEMM erstellt 			wurde
angle_magnet:		    Bogenlänge, die ein Magnet einnimmt (Rad: 70°, Ax: 60°)
angle_magnet_space:	Bogenlänge zwischen zweier Magneten (Rad:20°, Ax: 30°)
angle_coil:			    ?? Bogenlänge des Kerns einer Spule, nur für Radial (20°)
angle_coil_space:		Bogenlänge zwischen den Spulen
rotor_r_inner:	  Radius bis zur Innenseite des Magneten (Rad: 35 mm, 
